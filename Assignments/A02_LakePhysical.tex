\documentclass[]{article}
\usepackage{lmodern}
\usepackage{amssymb,amsmath}
\usepackage{ifxetex,ifluatex}
\usepackage{fixltx2e} % provides \textsubscript
\ifnum 0\ifxetex 1\fi\ifluatex 1\fi=0 % if pdftex
  \usepackage[T1]{fontenc}
  \usepackage[utf8]{inputenc}
\else % if luatex or xelatex
  \ifxetex
    \usepackage{mathspec}
  \else
    \usepackage{fontspec}
  \fi
  \defaultfontfeatures{Ligatures=TeX,Scale=MatchLowercase}
\fi
% use upquote if available, for straight quotes in verbatim environments
\IfFileExists{upquote.sty}{\usepackage{upquote}}{}
% use microtype if available
\IfFileExists{microtype.sty}{%
\usepackage{microtype}
\UseMicrotypeSet[protrusion]{basicmath} % disable protrusion for tt fonts
}{}
\usepackage[margin=2.54cm]{geometry}
\usepackage{hyperref}
\hypersetup{unicode=true,
            pdftitle={Assignment 2: Physical Properties of Lakes},
            pdfauthor={Jake Greif},
            pdfborder={0 0 0},
            breaklinks=true}
\urlstyle{same}  % don't use monospace font for urls
\usepackage{color}
\usepackage{fancyvrb}
\newcommand{\VerbBar}{|}
\newcommand{\VERB}{\Verb[commandchars=\\\{\}]}
\DefineVerbatimEnvironment{Highlighting}{Verbatim}{commandchars=\\\{\}}
% Add ',fontsize=\small' for more characters per line
\usepackage{framed}
\definecolor{shadecolor}{RGB}{248,248,248}
\newenvironment{Shaded}{\begin{snugshade}}{\end{snugshade}}
\newcommand{\AlertTok}[1]{\textcolor[rgb]{0.94,0.16,0.16}{#1}}
\newcommand{\AnnotationTok}[1]{\textcolor[rgb]{0.56,0.35,0.01}{\textbf{\textit{#1}}}}
\newcommand{\AttributeTok}[1]{\textcolor[rgb]{0.77,0.63,0.00}{#1}}
\newcommand{\BaseNTok}[1]{\textcolor[rgb]{0.00,0.00,0.81}{#1}}
\newcommand{\BuiltInTok}[1]{#1}
\newcommand{\CharTok}[1]{\textcolor[rgb]{0.31,0.60,0.02}{#1}}
\newcommand{\CommentTok}[1]{\textcolor[rgb]{0.56,0.35,0.01}{\textit{#1}}}
\newcommand{\CommentVarTok}[1]{\textcolor[rgb]{0.56,0.35,0.01}{\textbf{\textit{#1}}}}
\newcommand{\ConstantTok}[1]{\textcolor[rgb]{0.00,0.00,0.00}{#1}}
\newcommand{\ControlFlowTok}[1]{\textcolor[rgb]{0.13,0.29,0.53}{\textbf{#1}}}
\newcommand{\DataTypeTok}[1]{\textcolor[rgb]{0.13,0.29,0.53}{#1}}
\newcommand{\DecValTok}[1]{\textcolor[rgb]{0.00,0.00,0.81}{#1}}
\newcommand{\DocumentationTok}[1]{\textcolor[rgb]{0.56,0.35,0.01}{\textbf{\textit{#1}}}}
\newcommand{\ErrorTok}[1]{\textcolor[rgb]{0.64,0.00,0.00}{\textbf{#1}}}
\newcommand{\ExtensionTok}[1]{#1}
\newcommand{\FloatTok}[1]{\textcolor[rgb]{0.00,0.00,0.81}{#1}}
\newcommand{\FunctionTok}[1]{\textcolor[rgb]{0.00,0.00,0.00}{#1}}
\newcommand{\ImportTok}[1]{#1}
\newcommand{\InformationTok}[1]{\textcolor[rgb]{0.56,0.35,0.01}{\textbf{\textit{#1}}}}
\newcommand{\KeywordTok}[1]{\textcolor[rgb]{0.13,0.29,0.53}{\textbf{#1}}}
\newcommand{\NormalTok}[1]{#1}
\newcommand{\OperatorTok}[1]{\textcolor[rgb]{0.81,0.36,0.00}{\textbf{#1}}}
\newcommand{\OtherTok}[1]{\textcolor[rgb]{0.56,0.35,0.01}{#1}}
\newcommand{\PreprocessorTok}[1]{\textcolor[rgb]{0.56,0.35,0.01}{\textit{#1}}}
\newcommand{\RegionMarkerTok}[1]{#1}
\newcommand{\SpecialCharTok}[1]{\textcolor[rgb]{0.00,0.00,0.00}{#1}}
\newcommand{\SpecialStringTok}[1]{\textcolor[rgb]{0.31,0.60,0.02}{#1}}
\newcommand{\StringTok}[1]{\textcolor[rgb]{0.31,0.60,0.02}{#1}}
\newcommand{\VariableTok}[1]{\textcolor[rgb]{0.00,0.00,0.00}{#1}}
\newcommand{\VerbatimStringTok}[1]{\textcolor[rgb]{0.31,0.60,0.02}{#1}}
\newcommand{\WarningTok}[1]{\textcolor[rgb]{0.56,0.35,0.01}{\textbf{\textit{#1}}}}
\usepackage{graphicx,grffile}
\makeatletter
\def\maxwidth{\ifdim\Gin@nat@width>\linewidth\linewidth\else\Gin@nat@width\fi}
\def\maxheight{\ifdim\Gin@nat@height>\textheight\textheight\else\Gin@nat@height\fi}
\makeatother
% Scale images if necessary, so that they will not overflow the page
% margins by default, and it is still possible to overwrite the defaults
% using explicit options in \includegraphics[width, height, ...]{}
\setkeys{Gin}{width=\maxwidth,height=\maxheight,keepaspectratio}
\IfFileExists{parskip.sty}{%
\usepackage{parskip}
}{% else
\setlength{\parindent}{0pt}
\setlength{\parskip}{6pt plus 2pt minus 1pt}
}
\setlength{\emergencystretch}{3em}  % prevent overfull lines
\providecommand{\tightlist}{%
  \setlength{\itemsep}{0pt}\setlength{\parskip}{0pt}}
\setcounter{secnumdepth}{0}
% Redefines (sub)paragraphs to behave more like sections
\ifx\paragraph\undefined\else
\let\oldparagraph\paragraph
\renewcommand{\paragraph}[1]{\oldparagraph{#1}\mbox{}}
\fi
\ifx\subparagraph\undefined\else
\let\oldsubparagraph\subparagraph
\renewcommand{\subparagraph}[1]{\oldsubparagraph{#1}\mbox{}}
\fi

%%% Use protect on footnotes to avoid problems with footnotes in titles
\let\rmarkdownfootnote\footnote%
\def\footnote{\protect\rmarkdownfootnote}

%%% Change title format to be more compact
\usepackage{titling}

% Create subtitle command for use in maketitle
\providecommand{\subtitle}[1]{
  \posttitle{
    \begin{center}\large#1\end{center}
    }
}

\setlength{\droptitle}{-2em}

  \title{Assignment 2: Physical Properties of Lakes}
    \pretitle{\vspace{\droptitle}\centering\huge}
  \posttitle{\par}
    \author{Jake Greif}
    \preauthor{\centering\large\emph}
  \postauthor{\par}
    \date{}
    \predate{}\postdate{}
  

\begin{document}
\maketitle

\hypertarget{overview}{%
\subsection{OVERVIEW}\label{overview}}

This exercise accompanies the lessons in Hydrologic Data Analysis on the
physical properties of lakes.

\hypertarget{directions}{%
\subsection{Directions}\label{directions}}

\begin{enumerate}
\def\labelenumi{\arabic{enumi}.}
\tightlist
\item
  Change ``Student Name'' on line 3 (above) with your name.
\item
  Work through the steps, \textbf{creating code and output} that fulfill
  each instruction.
\item
  Be sure to \textbf{answer the questions} in this assignment document.
\item
  When you have completed the assignment, \textbf{Knit} the text and
  code into a single PDF file.
\item
  After Knitting, submit the completed exercise (PDF file) to the
  dropbox in Sakai. Add your last name into the file name (e.g.,
  ``Salk\_A02\_LakePhysical.Rmd'') prior to submission.
\end{enumerate}

The completed exercise is due on 11 September 2019 at 9:00 am.

\hypertarget{setup}{%
\subsection{Setup}\label{setup}}

\begin{enumerate}
\def\labelenumi{\arabic{enumi}.}
\tightlist
\item
  Verify your working directory is set to the R project file,
\item
  Load the tidyverse, lubridate, and cowplot packages
\item
  Import the NTL-LTER physical lake dataset and set the date column to
  the date format
\item
  Set your ggplot theme (can be theme\_classic or something else)
\end{enumerate}

\begin{Shaded}
\begin{Highlighting}[]
\CommentTok{# Check Working Directory}
\KeywordTok{getwd}\NormalTok{()}
\end{Highlighting}
\end{Shaded}

\begin{verbatim}
## [1] "/Users/jakegreif/Duke/Fall_2019/Hydrologic_Data_Analysis"
\end{verbatim}

\begin{Shaded}
\begin{Highlighting}[]
\CommentTok{# Load Packages}
\KeywordTok{library}\NormalTok{(tidyverse)}
\KeywordTok{library}\NormalTok{(lubridate)}
\KeywordTok{library}\NormalTok{(cowplot)}

\CommentTok{# Import Data}
\NormalTok{NTLdata <-}\StringTok{ }\KeywordTok{read.csv}\NormalTok{(}\StringTok{"./Data/Raw/NTL-LTER_Lake_ChemistryPhysics_Raw.csv"}\NormalTok{)}

\CommentTok{# Set ggplot Theme}
\NormalTok{mytheme <-}\StringTok{ }\KeywordTok{theme_classic}\NormalTok{()}

\KeywordTok{theme_set}\NormalTok{(mytheme)}
\end{Highlighting}
\end{Shaded}

\hypertarget{creating-and-analyzing-lake-temperature-profiles}{%
\subsection{Creating and analyzing lake temperature
profiles}\label{creating-and-analyzing-lake-temperature-profiles}}

\hypertarget{single-lake-multiple-dates}{%
\subsubsection{Single lake, multiple
dates}\label{single-lake-multiple-dates}}

\begin{enumerate}
\def\labelenumi{\arabic{enumi}.}
\setcounter{enumi}{4}
\tightlist
\item
  Choose either Peter or Tuesday Lake. Create a new data frame that
  wrangles the full data frame so that it only includes that lake during
  two different years (one year from the early part of the dataset and
  one year from the late part of the dataset).
\end{enumerate}

\begin{Shaded}
\begin{Highlighting}[]
\CommentTok{# Change sampledate to Date}
\NormalTok{NTLdata}\OperatorTok{$}\NormalTok{sampledate <-}\StringTok{ }\KeywordTok{as.Date}\NormalTok{(NTLdata}\OperatorTok{$}\NormalTok{sampledate, }\StringTok{"%m/%d/%y"}\NormalTok{)}

\CommentTok{# Make Tuesday Lake data frame}
\NormalTok{Tuesdaydata <-}\StringTok{ }\KeywordTok{filter}\NormalTok{(NTLdata, lakename }\OperatorTok{==}\StringTok{ "Tuesday Lake"}\NormalTok{)}

\CommentTok{# Trim Tuesday data frame to include only 1993 and 2016}
\NormalTok{Tues.data.skinny <-}\StringTok{ }\KeywordTok{filter}\NormalTok{(Tuesdaydata, year4 }\OperatorTok{==}\StringTok{ }\DecValTok{1993} \OperatorTok{|}\StringTok{ }\NormalTok{year4 }\OperatorTok{==}\StringTok{ }\DecValTok{2016}\NormalTok{)}

\CommentTok{# Make 2 separate data frames for each year}
\NormalTok{Tues}\FloatTok{.1993}\NormalTok{ <-}\StringTok{ }\KeywordTok{filter}\NormalTok{(Tues.data.skinny, year4 }\OperatorTok{==}\StringTok{ }\DecValTok{1993}\NormalTok{)}
\NormalTok{Tues}\FloatTok{.2016}\NormalTok{ <-}\StringTok{ }\KeywordTok{filter}\NormalTok{(Tues.data.skinny, year4 }\OperatorTok{==}\StringTok{ }\DecValTok{2016}\NormalTok{)}
\end{Highlighting}
\end{Shaded}

\begin{enumerate}
\def\labelenumi{\arabic{enumi}.}
\setcounter{enumi}{5}
\tightlist
\item
  Create three graphs: (1) temperature profiles for the early year, (2)
  temperature profiles for the late year, and (3) a \texttt{plot\_grid}
  of the two graphs together. Choose \texttt{geom\_point} and color your
  points by date.
\end{enumerate}

Remember to edit your graphs so they follow good data visualization
practices.

\begin{Shaded}
\begin{Highlighting}[]
\NormalTok{Tempprofiles1993 <-}\StringTok{ }
\StringTok{  }\KeywordTok{ggplot}\NormalTok{(Tues}\FloatTok{.1993}\NormalTok{, }\KeywordTok{aes}\NormalTok{(}\DataTypeTok{x =}\NormalTok{ temperature_C,}\DataTypeTok{y =}\NormalTok{ depth, }\DataTypeTok{color =}\NormalTok{ daynum)) }\OperatorTok{+}
\StringTok{  }\KeywordTok{geom_point}\NormalTok{() }\OperatorTok{+}
\StringTok{  }\KeywordTok{scale_y_reverse}\NormalTok{() }\OperatorTok{+}
\StringTok{  }\KeywordTok{scale_x_continuous}\NormalTok{(}\DataTypeTok{position =} \StringTok{"top"}\NormalTok{, }\DataTypeTok{limits =} \KeywordTok{c}\NormalTok{(}\DecValTok{4}\NormalTok{,}\DecValTok{25}\NormalTok{)) }\OperatorTok{+}
\StringTok{  }\KeywordTok{scale_color_viridis_c}\NormalTok{(}\DataTypeTok{end =} \FloatTok{0.8}\NormalTok{, }\DataTypeTok{option =} \StringTok{"magma"}\NormalTok{) }\OperatorTok{+}
\StringTok{  }\KeywordTok{labs}\NormalTok{(}\DataTypeTok{x =} \KeywordTok{expression}\NormalTok{(}\StringTok{"1993 Temperature "}\NormalTok{(degree}\OperatorTok{*}\NormalTok{C)), }\DataTypeTok{y =} \StringTok{"Depth (m)"}\NormalTok{) }\OperatorTok{+}
\StringTok{  }\KeywordTok{theme}\NormalTok{(}\DataTypeTok{legend.position =} \StringTok{"none"}\NormalTok{)}
\KeywordTok{print}\NormalTok{(Tempprofiles1993)}
\end{Highlighting}
\end{Shaded}

\includegraphics{A02_LakePhysical_files/figure-latex/unnamed-chunk-3-1.pdf}

\begin{Shaded}
\begin{Highlighting}[]
\NormalTok{Tempprofiles2016 <-}\StringTok{ }
\StringTok{  }\KeywordTok{ggplot}\NormalTok{(Tues}\FloatTok{.2016}\NormalTok{, }\KeywordTok{aes}\NormalTok{(}\DataTypeTok{x =}\NormalTok{ temperature_C, }\DataTypeTok{y =}\NormalTok{ depth, }\DataTypeTok{color =}\NormalTok{ daynum)) }\OperatorTok{+}
\StringTok{  }\KeywordTok{geom_point}\NormalTok{() }\OperatorTok{+}
\StringTok{  }\KeywordTok{scale_y_reverse}\NormalTok{() }\OperatorTok{+}
\StringTok{  }\KeywordTok{scale_x_continuous}\NormalTok{(}\DataTypeTok{position =} \StringTok{"top"}\NormalTok{) }\OperatorTok{+}
\StringTok{  }\KeywordTok{scale_color_viridis_c}\NormalTok{(}\DataTypeTok{end =} \FloatTok{0.8}\NormalTok{, }\DataTypeTok{option =} \StringTok{"magma"}\NormalTok{) }\OperatorTok{+}
\StringTok{  }\KeywordTok{labs}\NormalTok{(}\DataTypeTok{x =} \KeywordTok{expression}\NormalTok{(}\StringTok{"2016 Temperature "}\NormalTok{(degree}\OperatorTok{*}\NormalTok{C)), }\DataTypeTok{y =} \StringTok{"Depth (m)"}\NormalTok{,}
       \DataTypeTok{color =} \StringTok{"Ordinal Day"}\NormalTok{) }\OperatorTok{+}
\StringTok{  }\KeywordTok{theme}\NormalTok{(}\DataTypeTok{axis.text.y =} \KeywordTok{element_blank}\NormalTok{(), }\DataTypeTok{axis.title.y =} \KeywordTok{element_blank}\NormalTok{())}
\KeywordTok{print}\NormalTok{(Tempprofiles2016)}
\end{Highlighting}
\end{Shaded}

\includegraphics{A02_LakePhysical_files/figure-latex/unnamed-chunk-3-2.pdf}

\begin{Shaded}
\begin{Highlighting}[]
\NormalTok{Temp.}\FloatTok{1993.2016}\NormalTok{ <-}\StringTok{ }
\StringTok{  }\KeywordTok{plot_grid}\NormalTok{(Tempprofiles1993, Tempprofiles2016, }
            \DataTypeTok{ncol =} \DecValTok{2}\NormalTok{, }\DataTypeTok{rel_widths =} \KeywordTok{c}\NormalTok{(}\FloatTok{0.9}\NormalTok{, }\DecValTok{1}\NormalTok{))}
\KeywordTok{print}\NormalTok{(Temp.}\FloatTok{1993.2016}\NormalTok{)}
\end{Highlighting}
\end{Shaded}

\includegraphics{A02_LakePhysical_files/figure-latex/unnamed-chunk-3-3.pdf}

\begin{enumerate}
\def\labelenumi{\arabic{enumi}.}
\setcounter{enumi}{6}
\tightlist
\item
  Interpret the stratification patterns in your graphs in light of
  seasonal trends. In addition, do you see differences between the two
  years?
\end{enumerate}

\begin{quote}
Temperatures increase through the year, but the magnitude of the
increase decreases with depth. Temperature essentially doesn't change
below 7.5 m, which is due to thermal stratification and a lack of mixing
in the summer. In 1993 the epilimnion was cooler throughout the summer
than in 2016, but the metalimnion warmed slightly in 1993 compared to
2016.
\end{quote}

\hypertarget{multiple-lakes-single-date}{%
\subsubsection{Multiple lakes, single
date}\label{multiple-lakes-single-date}}

\begin{enumerate}
\def\labelenumi{\arabic{enumi}.}
\setcounter{enumi}{7}
\tightlist
\item
  On July 25, 26, and 27 in 2016, all three lakes (Peter, Paul, and
  Tuesday) were sampled. Wrangle your data frame to include just these
  three dates.
\end{enumerate}

\begin{Shaded}
\begin{Highlighting}[]
\NormalTok{July}\FloatTok{.2016}\NormalTok{ <-}\StringTok{ }\KeywordTok{filter}\NormalTok{(NTLdata, lakename }\OperatorTok{==}\StringTok{ "Peter Lake"} \OperatorTok{|}
\StringTok{                      }\NormalTok{lakename }\OperatorTok{==}\StringTok{ "Paul Lake"} \OperatorTok{|}
\StringTok{                      }\NormalTok{lakename }\OperatorTok{==}\StringTok{ "Tuesday Lake"}\NormalTok{)}
\NormalTok{July}\FloatTok{.2016}\NormalTok{ <-}\StringTok{ }\KeywordTok{filter}\NormalTok{(July}\FloatTok{.2016}\NormalTok{, sampledate }\OperatorTok{==}\StringTok{ "2016-07-25"} \OperatorTok{|}
\StringTok{                      }\NormalTok{sampledate }\OperatorTok{==}\StringTok{ "2016-07-26"} \OperatorTok{|}
\StringTok{                      }\NormalTok{sampledate }\OperatorTok{==}\StringTok{ "2016-07-27"}\NormalTok{)}
\end{Highlighting}
\end{Shaded}

\begin{enumerate}
\def\labelenumi{\arabic{enumi}.}
\setcounter{enumi}{8}
\tightlist
\item
  Plot a profile line graph of temperature by depth, one line per lake.
  Each lake can be designated by a separate color.
\end{enumerate}

\begin{Shaded}
\begin{Highlighting}[]
\CommentTok{# Separate data frame by lake}
\NormalTok{Peter.July}\FloatTok{.2016}\NormalTok{ <-}\StringTok{ }\KeywordTok{filter}\NormalTok{(July}\FloatTok{.2016}\NormalTok{, lakename }\OperatorTok{==}\StringTok{ "Peter Lake"}\NormalTok{)}
\NormalTok{Paul.July}\FloatTok{.2016}\NormalTok{ <-}\StringTok{ }\KeywordTok{filter}\NormalTok{(July}\FloatTok{.2016}\NormalTok{, lakename }\OperatorTok{==}\StringTok{ "Paul Lake"}\NormalTok{)}
\NormalTok{Tues.July}\FloatTok{.2016}\NormalTok{ <-}\StringTok{ }\KeywordTok{filter}\NormalTok{(July}\FloatTok{.2016}\NormalTok{, lakename }\OperatorTok{==}\StringTok{ "Tuesday Lake"}\NormalTok{)}

\CommentTok{# Plot temp profile of each lake}
\NormalTok{PeterTempProfileJuly2016 <-}\StringTok{ }
\StringTok{  }\KeywordTok{ggplot}\NormalTok{(Peter.July}\FloatTok{.2016}\NormalTok{, }\KeywordTok{aes}\NormalTok{(}\DataTypeTok{x =}\NormalTok{ temperature_C, }\DataTypeTok{y =}\NormalTok{ depth)) }\OperatorTok{+}
\StringTok{  }\KeywordTok{geom_line}\NormalTok{(}\DataTypeTok{color =} \StringTok{"#081d58"}\NormalTok{) }\OperatorTok{+}
\StringTok{  }\KeywordTok{geom_vline}\NormalTok{(}\DataTypeTok{xintercept =} \DecValTok{4}\NormalTok{, }\DataTypeTok{lty =} \DecValTok{2}\NormalTok{) }\OperatorTok{+}
\StringTok{  }\KeywordTok{scale_y_reverse}\NormalTok{(}\DataTypeTok{breaks =} \KeywordTok{c}\NormalTok{(}\DecValTok{0}\NormalTok{, }\DecValTok{2}\NormalTok{, }\DecValTok{4}\NormalTok{, }\DecValTok{6}\NormalTok{, }\DecValTok{8}\NormalTok{, }\DecValTok{10}\NormalTok{, }\DecValTok{12}\NormalTok{)) }\OperatorTok{+}
\StringTok{  }\KeywordTok{scale_x_continuous}\NormalTok{(}\DataTypeTok{position =} \StringTok{"top"}\NormalTok{, }\DataTypeTok{limits =} \KeywordTok{c}\NormalTok{(}\DecValTok{0}\NormalTok{, }\DecValTok{30}\NormalTok{)) }\OperatorTok{+}
\StringTok{  }\KeywordTok{labs}\NormalTok{(}\DataTypeTok{x =} \KeywordTok{expression}\NormalTok{(}\StringTok{"Temperature "}\NormalTok{(degree}\OperatorTok{*}\NormalTok{C)), }\DataTypeTok{y =} \StringTok{"Depth (m)"}\NormalTok{)}
\KeywordTok{print}\NormalTok{(PeterTempProfileJuly2016)}
\end{Highlighting}
\end{Shaded}

\includegraphics{A02_LakePhysical_files/figure-latex/unnamed-chunk-5-1.pdf}

\begin{Shaded}
\begin{Highlighting}[]
\NormalTok{PaulTempProfileJuly2016 <-}\StringTok{ }
\StringTok{  }\NormalTok{PeterTempProfileJuly2016 }\OperatorTok{+}
\StringTok{  }\KeywordTok{geom_line}\NormalTok{(}\DataTypeTok{data =}\NormalTok{ Paul.July}\FloatTok{.2016}\NormalTok{, }\KeywordTok{aes}\NormalTok{(}\DataTypeTok{x =}\NormalTok{ temperature_C, }\DataTypeTok{y =}\NormalTok{ depth), }
            \DataTypeTok{color =} \StringTok{"#1d91c0"}\NormalTok{)}
\KeywordTok{print}\NormalTok{(PaulTempProfileJuly2016)}
\end{Highlighting}
\end{Shaded}

\includegraphics{A02_LakePhysical_files/figure-latex/unnamed-chunk-5-2.pdf}

\begin{Shaded}
\begin{Highlighting}[]
\NormalTok{TuesTempProfileJuly2016 <-}\StringTok{ }
\StringTok{  }\NormalTok{PaulTempProfileJuly2016 }\OperatorTok{+}
\StringTok{  }\KeywordTok{geom_line}\NormalTok{(}\DataTypeTok{data =}\NormalTok{ Tues.July}\FloatTok{.2016}\NormalTok{, }\KeywordTok{aes}\NormalTok{(}\DataTypeTok{x =}\NormalTok{ temperature_C, }\DataTypeTok{y =}\NormalTok{ depth), }
            \DataTypeTok{color =} \StringTok{"#225ea8"}\NormalTok{)}
\KeywordTok{print}\NormalTok{(TuesTempProfileJuly2016)}
\end{Highlighting}
\end{Shaded}

\includegraphics{A02_LakePhysical_files/figure-latex/unnamed-chunk-5-3.pdf}

\begin{enumerate}
\def\labelenumi{\arabic{enumi}.}
\setcounter{enumi}{9}
\tightlist
\item
  What is the depth range of the epilimnion in each lake? The
  thermocline? The hypolimnion?
\end{enumerate}

\begin{quote}
The range of the epilimnion is from the surface to 3 meters. The range
of the thermocline is from 2 meters to approximately 9 meters. And the
range of the hypolmnion is from about 6 meters to the lake bed.
\end{quote}

\hypertarget{trends-in-surface-temperatures-over-time.}{%
\subsection{Trends in surface temperatures over
time.}\label{trends-in-surface-temperatures-over-time.}}

\begin{enumerate}
\def\labelenumi{\arabic{enumi}.}
\setcounter{enumi}{10}
\tightlist
\item
  Run the same analyses we ran in class to determine if surface lake
  temperatures for a given month have increased over time (``Long-term
  change in temperature'' section of day 4 lesson in its entirety), this
  time for either Peter or Tuesday Lake.
\end{enumerate}

\begin{Shaded}
\begin{Highlighting}[]
\CommentTok{# 1. Add Month Column}
\NormalTok{Tuesdaydata}\OperatorTok{$}\NormalTok{Month <-}\StringTok{ }\KeywordTok{substring}\NormalTok{(Tuesdaydata}\OperatorTok{$}\NormalTok{sampledate, }\DecValTok{7}\NormalTok{,}\DecValTok{7}\NormalTok{)}

\CommentTok{# 2. Filter data frame}
\NormalTok{Tues}\FloatTok{.0}\NormalTok{ <-}\StringTok{ }\KeywordTok{filter}\NormalTok{(Tuesdaydata, depth }\OperatorTok{==}\StringTok{ }\DecValTok{0}\NormalTok{)}
\NormalTok{Tues.summer <-}\StringTok{ }\KeywordTok{filter}\NormalTok{(Tues}\FloatTok{.0}\NormalTok{, Month }\OperatorTok{==}\StringTok{ }\KeywordTok{c}\NormalTok{(}\DecValTok{5}\NormalTok{,}\DecValTok{6}\NormalTok{,}\DecValTok{7}\NormalTok{,}\DecValTok{8}\NormalTok{))}
\NormalTok{Tues.summer}\OperatorTok{$}\NormalTok{Month[Tues.summer}\OperatorTok{$}\NormalTok{Month }\OperatorTok{==}\StringTok{ }\DecValTok{5}\NormalTok{] <-}\StringTok{ "May"}
\NormalTok{Tues.summer}\OperatorTok{$}\NormalTok{Month[Tues.summer}\OperatorTok{$}\NormalTok{Month }\OperatorTok{==}\StringTok{ }\DecValTok{6}\NormalTok{] <-}\StringTok{ "June"}
\NormalTok{Tues.summer}\OperatorTok{$}\NormalTok{Month[Tues.summer}\OperatorTok{$}\NormalTok{Month }\OperatorTok{==}\StringTok{ }\DecValTok{7}\NormalTok{] <-}\StringTok{ "July"}
\NormalTok{Tues.summer}\OperatorTok{$}\NormalTok{Month[Tues.summer}\OperatorTok{$}\NormalTok{Month }\OperatorTok{==}\StringTok{ }\DecValTok{8}\NormalTok{] <-}\StringTok{ "August"}

\CommentTok{# 3. Create 4 Separate DF}
\NormalTok{Tues5 <-}\StringTok{ }\KeywordTok{filter}\NormalTok{(Tues.summer, Month }\OperatorTok{==}\StringTok{ "May"}\NormalTok{)}
\NormalTok{Tues6 <-}\StringTok{ }\KeywordTok{filter}\NormalTok{(Tues.summer, Month }\OperatorTok{==}\StringTok{ "June"}\NormalTok{)}
\NormalTok{Tues7 <-}\StringTok{ }\KeywordTok{filter}\NormalTok{(Tues.summer, Month }\OperatorTok{==}\StringTok{ "July"}\NormalTok{)}
\NormalTok{Tues8 <-}\StringTok{ }\KeywordTok{filter}\NormalTok{(Tues.summer, Month }\OperatorTok{==}\StringTok{ "August"}\NormalTok{)}

\CommentTok{# 4. Run Linear Regression}
\NormalTok{Tues5.lm <-}\StringTok{ }\KeywordTok{lm}\NormalTok{(}\DataTypeTok{data =}\NormalTok{ Tues5, temperature_C }\OperatorTok{~}\StringTok{ }\NormalTok{year4)}
\KeywordTok{summary}\NormalTok{(Tues5.lm)}
\end{Highlighting}
\end{Shaded}

\begin{verbatim}
## 
## Call:
## lm(formula = temperature_C ~ year4, data = Tues5)
## 
## Residuals:
##     Min      1Q  Median      3Q     Max 
## -4.2491 -2.1410  0.1716  2.1112  5.3004 
## 
## Coefficients:
##               Estimate Std. Error t value Pr(>|t|)
## (Intercept)  36.003218 171.065481   0.210    0.837
## year4        -0.009912   0.085786  -0.116    0.910
## 
## Residual standard error: 3.039 on 11 degrees of freedom
## Multiple R-squared:  0.001212,   Adjusted R-squared:  -0.08959 
## F-statistic: 0.01335 on 1 and 11 DF,  p-value: 0.9101
\end{verbatim}

\begin{Shaded}
\begin{Highlighting}[]
\NormalTok{Tues6.lm <-}\StringTok{ }\KeywordTok{lm}\NormalTok{(}\DataTypeTok{data =}\NormalTok{ Tues6, temperature_C }\OperatorTok{~}\StringTok{ }\NormalTok{year4)}
\KeywordTok{summary}\NormalTok{(Tues6.lm)}
\end{Highlighting}
\end{Shaded}

\begin{verbatim}
## 
## Call:
## lm(formula = temperature_C ~ year4, data = Tues6)
## 
## Residuals:
##     Min      1Q  Median      3Q     Max 
## -5.4907 -1.9893  0.0872  1.4401  4.4524 
## 
## Coefficients:
##              Estimate Std. Error t value Pr(>|t|)
## (Intercept) 101.60736  111.36885   0.912    0.372
## year4        -0.04051    0.05577  -0.726    0.476
## 
## Residual standard error: 2.722 on 20 degrees of freedom
## Multiple R-squared:  0.02571,    Adjusted R-squared:  -0.02301 
## F-statistic: 0.5277 on 1 and 20 DF,  p-value: 0.476
\end{verbatim}

\begin{Shaded}
\begin{Highlighting}[]
\NormalTok{Tues7.lm <-}\StringTok{ }\KeywordTok{lm}\NormalTok{(}\DataTypeTok{data =}\NormalTok{ Tues7, temperature_C }\OperatorTok{~}\StringTok{ }\NormalTok{year4)}
\KeywordTok{summary}\NormalTok{(Tues7.lm)}
\end{Highlighting}
\end{Shaded}

\begin{verbatim}
## 
## Call:
## lm(formula = temperature_C ~ year4, data = Tues7)
## 
## Residuals:
##     Min      1Q  Median      3Q     Max 
## -3.0410 -1.1147 -0.1347  1.2054  3.1386 
## 
## Coefficients:
##              Estimate Std. Error t value Pr(>|t|)
## (Intercept)  9.664936  62.268171   0.155    0.878
## year4        0.006692   0.031164   0.215    0.832
## 
## Residual standard error: 1.597 on 18 degrees of freedom
## Multiple R-squared:  0.002555,   Adjusted R-squared:  -0.05286 
## F-statistic: 0.04611 on 1 and 18 DF,  p-value: 0.8324
\end{verbatim}

\begin{Shaded}
\begin{Highlighting}[]
\NormalTok{Tues8.lm <-}\StringTok{ }\KeywordTok{lm}\NormalTok{(}\DataTypeTok{data =}\NormalTok{ Tues8, temperature_C }\OperatorTok{~}\StringTok{ }\NormalTok{year4)}
\KeywordTok{summary}\NormalTok{(Tues8.lm)}
\end{Highlighting}
\end{Shaded}

\begin{verbatim}
## 
## Call:
## lm(formula = temperature_C ~ year4, data = Tues8)
## 
## Residuals:
##     Min      1Q  Median      3Q     Max 
## -3.3283 -1.2906 -0.1875  1.4390  3.5717 
## 
## Coefficients:
##              Estimate Std. Error t value Pr(>|t|)
## (Intercept) -84.29885   83.01433  -1.015    0.324
## year4         0.05308    0.04159   1.276    0.219
## 
## Residual standard error: 1.881 on 17 degrees of freedom
## Multiple R-squared:  0.08744,    Adjusted R-squared:  0.03376 
## F-statistic: 1.629 on 1 and 17 DF,  p-value: 0.219
\end{verbatim}

\begin{Shaded}
\begin{Highlighting}[]
\CommentTok{# 5. Calculate how many degrees the lake has warmed (for sig. trends)}

\CommentTok{# There are no signifcant surface temperature trends in Tuesday Lake.}

\CommentTok{# 6. Plot surface temps by date and facet by month}
\NormalTok{TuesSurfTemps <-}
\StringTok{  }\KeywordTok{ggplot}\NormalTok{(Tues.summer, }\KeywordTok{aes}\NormalTok{(}\DataTypeTok{x =}\NormalTok{ sampledate, }\DataTypeTok{y =}\NormalTok{ temperature_C,}
            \DataTypeTok{color =}\NormalTok{ Month)) }\OperatorTok{+}
\StringTok{  }\KeywordTok{labs}\NormalTok{(}\DataTypeTok{x =} \StringTok{"Year"}\NormalTok{, }\DataTypeTok{y =} \KeywordTok{expression}\NormalTok{(}\StringTok{"Temperature "}\NormalTok{(degree}\OperatorTok{*}\NormalTok{C))) }\OperatorTok{+}
\StringTok{  }\KeywordTok{geom_smooth}\NormalTok{(}\DataTypeTok{se =} \OtherTok{FALSE}\NormalTok{, }\DataTypeTok{method =}\NormalTok{ lm, }\DataTypeTok{color =} \StringTok{"black"}\NormalTok{, }\DataTypeTok{size =} \FloatTok{0.8}\NormalTok{) }\OperatorTok{+}
\StringTok{  }\KeywordTok{geom_point}\NormalTok{() }\OperatorTok{+}
\StringTok{  }\KeywordTok{facet_wrap}\NormalTok{(}\KeywordTok{vars}\NormalTok{(Month), }\DataTypeTok{nrow =} \DecValTok{2}\NormalTok{)}
\KeywordTok{print}\NormalTok{(TuesSurfTemps)}
\end{Highlighting}
\end{Shaded}

\includegraphics{A02_LakePhysical_files/figure-latex/unnamed-chunk-6-1.pdf}

\begin{enumerate}
\def\labelenumi{\arabic{enumi}.}
\setcounter{enumi}{11}
\tightlist
\item
  How do your results compare to those we found in class for Paul Lake?
  Do similar trends exist for both lakes?
\end{enumerate}

\begin{quote}
There are no significant trends in Tuesday Lake, compared to Paul Lake
which had significant increasing trends in two months.
\end{quote}


\end{document}
